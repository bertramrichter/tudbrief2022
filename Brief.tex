\documentclass[twoside]{tudbrief2022}
\usepackage[ngerman]{babel}		% for hyphenation and date format

% Absenderfeld
\ABSName{Name des Absenders}
\ABSStrasse{Absenderstraße 1}
\ABSPLZ{01234}
\ABSOrt{Absenderort}
\ABSPhone{+49~123~123456789}
\ABSMobile{+49~123~123456789}
\ABSFax{+49~123~123456789}
\ABSMail{vorname.nachname@tu-dresden.de}
\ABSWeb{tu-dresden.de/cd}
\ABSFunktion{Wissenschaftlicher Mitarbeiter}
\ABSInfo{Zusätzliche Informationen über den Absender}
% Adressfeld
\ZViii{elektronische Freimachungsvermerke}
\ZVii{Vorausfügung}
\ZVi{z.\,B.\ Einschreiben}
\AZFirma{Firma}
\AZAnrede{Frau/Herr}
\AZName{Name des Addressaten}
\AZStrasse{Zielstraße 1}
\AZPLZOrt{12345 Zielort}
\AZLand{Deutschland}
% Betreff
\Betreff{Betreff}
% Aussehen
\institut{Absendende Struktureinheit der TU Dresden}
\logoeinrichtung{./template/Logo_Einrichtung.pdf}
%\Briefpapier{}
\Signatur{Unterschrift.png}


\begin{document}
	\maketitle{}
	\Anrede{Sehr geehrte Mitglieder und Angehörige der TU Dresden}
	
	dies ist die \emph{Vorlage für einen allgemeinen Brief}.
	Diese Vorlage umfasst eine Reihe von Voreinstellungen, welche dazu beitragen, den Brief schnell und effizient zu formatieren und bei Bedarf als barrierefreies Dokument zu exportieren.
	Über Links im Text haben wir Ihnen Webseiten oder Dokumente mit nützlichen Anleitungen hinterlegt.
	\textcolor{red}{Bitte lesen Sie erst beide Seiten, bevor Sie mit der Anpassung der Vorlage beginnen.}
	
	\section*{Dokumentenstruktur}
	Das Dokument wird automatisch gesetzt. In der Dokumentation (\verb|tudbrief2022.pdf|) sind sämtliche Optionen aufgeführt.
	
	\Gruss{mit freundlichem Gruß}
	
	% Post Scriptum
	%PS: Ich bin bis März nur telefonisch erreichbar.
	
	% Weitere PDFs können automatisch angefügt werden, z.B. Anhänge.
	%\includepdf[pages=-,openright]{pfad/zu/weiteren/pdfs/dokument.pdf}
	% Pfad ist relativ zu dieser tex-Datei. Mit .. ein Verzeichnis hoch.
	% Der pages-Parameter spezifiziert welche Seiten eingefügt werden.
	% Beispiele:
	% pages=-				alle Seiten
	% pages={1-4}			Seite 1-4
	% pages={1,4,5}			Seite 1, 4 und 5
	% pages={3,{},8-11,15}	Seite 3, leere Seite, Seite 8-11 und Seite 15
	% Der openright-Parameter startet die Anlagen auf ungerader (rechter) Seite, d.h. notfalls wird eine leere Seite
	% eingefügt. Im doppelseitigem Druck wird dadurch besser zwischen Brief und Anlage getrennt. Für einseitigen Druck
	% entfernen.
\end{document}
